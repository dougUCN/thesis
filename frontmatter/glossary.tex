\glsaddkey
    {value}% new "value" key for symbols table
    {}% default value if "value" isn't used in \newglossaryentry
    {\glsentryvalue}% analogous to \glsentrytext
    {\Glsentryvalue}% analogous to \Glsentrytext
    {\glsvalue}% analogous to \glstext
    {\Glsvalue}% analogous to \Glstext
    {\GLSvalue}% analogous to \GLStext

\newglossary*{doNotPrint}{Do Not Print} % For glossary entries we don't want printed

% https://tex.stackexchange.com/questions/588051/glossary-entries-error
% protect non-robust commands
\newcommand{\glsvv}[1]{\protect\vv{#1}}

% https://tex.stackexchange.com/questions/107536/how-do-i-sort-two-glossaries-list-of-acronyms-and-list-of-symbols-differently
\makenoidxglossaries

%%%%%%%%%%%%%%%%%%%%%%%%%%%%%%%%%%%
% List of symbols
% "value" field is customized for the glossary style defined in this file
%%%%%%%%%%%%%%%%%%%%%%%%%%%%%%%%%%%

\newglossaryentry{T_fp}{
    name=\ensuremath{T_\text{fp}},
    description={Ramsey method free precession period},
}

\newglossaryentry{alpha}{
    name=\ensuremath{\alpha},
    description={Spin contrast of a Ramsey fringe},
}

\newglossaryentry{d_n}{
    name=\ensuremath{d_\text{n}},
    description={Neutron electric dipole moment},
}

\newglossaryentry{adiab}{
    name=\ensuremath{\eta},
    description={Adiabaticity},
}

\newglossaryentry{hbar}{
    name=\ensuremath{\hbar},
    description={reduced Planck's constant},
    value={$6.582\,119\,569...\times 10^{-16}\text{ eV s}$}
}

\newglossaryentry{mu_n}{
    name=\ensuremath{\mu_\text{n}},
    description={Neutron magnetic moment},
    value={$-9.662\,365\,1(23)\times10^{-27}\text{ J}^{-1}\text{T}^{-1}$
          \\ & & $60.307\,739(15)\text{ neV T}^{-1}$}
}

\newglossaryentry{gamma_n}{
    name=\ensuremath{\gamma_\text{n}},
    description={Neutron gyromagnetic ratio $2|\mu_\text{n}|/\hbar$},
    value={$-1.832\,471\,71(43)\times10^8\text{rad s}^{-1}\text{T}^{-1}$
          \\ & & $29.164\,693\,1(69)\text{ MHz T}^{-1}$}
}

\newglossaryentry{m_n}{
    name=\ensuremath{m_\text{n}},
    description={Neutron mass},
    value={$1.674\,927\,498\,04(95)\times 10^{-27}\text{ kg}$
         \\ & & $939.565\,420\,52(54)\text{ MeV }c^{-2}$}
}

\newglossaryentry{mg}{
    name=\ensuremath{m_\text{n}g_0},
    description={Gravity acel. $g_0$ on neutron mass $m_\text{n}$},
    value={$102.519\,445\,56(64)\text{ neV m}^{-1}$}
}

\newglossaryentry{tau_n}{
    name=\ensuremath{\tau_\text{n}},
    description={Neutron lifetime},
    value={\ensuremath{878.4(5)\text{ s}}}
}

\newglossaryentry{q_n}{
    name=\ensuremath{q_\text{n}},
    description={Neutron charge limit},
    value={\ensuremath{(-0.4 \pm 1.1 ) \times 10^{-21} e}}
}

%%%%%%%%%%%%%%%%%%%%%%%%%%%%%%%%%%%
% List of symbols not printed in glossary
%%%%%%%%%%%%%%%%%%%%%%%%%%%%%%%%%%%

\newglossaryentry{bField}{
    name=\ensuremath{\glsvv{B}},
    description={Magnetic field},
    type=doNotPrint,
}

\newglossaryentry{lorentz}{
    name=\ensuremath{\Gamma},
    description={Relativistic Lorentz factor in the lab frame, $1/\sqrt{1-v^2/c^2}$},
    type=doNotPrint,
}

\newglossaryentry{vspec}{
    name=\ensuremath{a},
    description={Exponent in velocity spectrum $v^a\,dv$},
    type=doNotPrint,
}

\newglossaryentry{rho_N}{
    name=\ensuremath{\rho_N},
    description={Number density},
    type=doNotPrint,
}

%%%%%%%%%%%%%%%%%%%%%%%%%%%%%%%%%%%
% Acronyms
%%%%%%%%%%%%%%%%%%%%%%%%%%%%%%%%%%%

\newacronym{lanl}{LANL}{Los Alamos National Laboratory}

\newacronym{nedm}{nEDM}{Neutron electric dipole moment}

\newacronym{edm}{EDM}{Electric dipole moment}

\newacronym{ucn}{UCN}{Ultracold neutron(s)}

\newacronym{rf}{RF}{Radio frequency}

\newacronym{cl}{CL}{Confidence level}

\newacronym{iu}{IU}{Indiana University Bloomington}

\newacronym{msr}{MSR}{Magnetically shielded room}

\newacronym{nmr}{NMR}{Nuclear magnetic resonance}

\newacronym{afp}{AFP}{Adiabatic fast passage}

\newacronym{fwhm}{FWHM}{Full width half maximum}

\newacronym{sm}{SM}{Standard Model}

\newacronym{bsm}{BSM}{Beyond Standard Model}

\newacronym{bau}{BAU}{Baryon matter/anti-matter asymmetry of the universe}

\newacronym{ss}{SS}{Stainless steel}

\newacronym{nip}{NiP}{Nickel phosphorus}

\newacronym{nimo}{NiMo}{Nickel molybdenum}

\newacronym{dps}{dPS}{Deuterated polystyrene}

\newacronym{dlc}{DLC}{Diamond-like carbon}

\newacronym{pm}{PM}{Polarizing magnet (Sec.~\ref{sec:PM_description})}

\newacronym{pmt}{PMT}{Photomultiplier tube}

\newacronym{sipm}{SiPM}{Silicon photomultiplier}

%%%%%%%%%%%%%%%%%%%%%%%%%%%%%%%%%%%
% Custom glossary style for symbols
%%%%%%%%%%%%%%%%%%%%%%%%%%%%%%%%%%%
% Documentation found in [Documented Code for glossaries](https://ctan.math.illinois.edu/macros/latex/contrib/glossaries/glossaries-code.pdf)
\newcommand*{\valuename}{{\bfseries{Value}} (Refs.~\cite{codata_2018, pdg2022})}% Name of the third column
\def\glsvaluewidth{0.7\hsize} % width of the third column
\newglossarystyle{symbolstyle}{%
    \renewenvironment{theglossary}%
    {%
        % Definition of header row
        \tablehead{\bfseries\entryname &
        \bfseries\descriptionname &
        \valuename \tabularnewline}% 
        \tabletail{}%
        \begin{supertabular}{lp{\glsdescwidth}p{\glsvaluewidth}}
    }%
    {\end{supertabular}}%

    \renewcommand*{\glossaryheader}{}%
    \renewcommand*{\glsgroupheading}[1]{}%
    \renewcommand{\glossentry}[2]{%
        \glsentryitem{##1}\glstarget{##1}{\glossentryname{##1}} &
        \glossentrydesc{##1} &
        \glsentryvalue{##1}\tabularnewline
    }%

    \ifglsnogroupskip
        \renewcommand*{\glsgroupskip}{}%
    \else
        \renewcommand*{\glsgroupskip}{& \tabularnewline}%
    \fi

    \renewcommand*{\entryname}{Symbol} % Change name of the first column
    \setlength{\glsdescwidth}{0.45\hsize} % change width of second column
}
