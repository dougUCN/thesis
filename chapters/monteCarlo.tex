%%%%%%%%%%%%%%%%%%%%%%%%%%%%%%%%%%%%%%%%%%

\chapter{Monte Carlo simulations of UCN transport and spin transport}\label{chap:simulations}

%%%%%%%%%%%%%%%%%%%%%%%%%%%%%%%%%%%%%%%%%%

Secs. blah and blah use PENTrack to run ucn transport simulations. The height differential problem... The spin asymmetry problem... Simulations were run on the IU Carbonate cluster... Sec. blah provides a demonstration of modeling transport into the cell with a simple 1D random walk. 

%%%%%%%%%%%%%%%%%%%%%%%%%%%%%%%%%%%%%%%%%%

\section{PENTrack}

%%%%%%%%%%%%%%%%%%%%%%%%%%%%%%%%%%%%%%%%%%

PENTrack \cite{schreyer_pentrack} is a simulation tool written in C++ for UCN experiments developed by Wolfgang Schreyer. It supports many excellent features, including:
%
\begin{itemize}
    \item Arbitrary experimental geometries defined by STL files
    \item Arbitrary analytical and numerical electromagnetic fields, with time dependence
    \item Definition of multiple material types via real and imaginary optical potentials~(Sec.~\ref{sec:ucn_matter_int})
    \item Diffuse reflection with Lambertian and micro roughness models (Sec.~\ref{sec:diffuse_reflections})
    \item Relativistic spin tracking with the BMT equation (Sec.~\ref{sec:BMT_equations}). Spin tracking of high speed particles such as electrons, protons, etc. are supported.
    \item Simulation of neutron decay and the tracking of resulting protons and electrons
    \item Simulation of xenon, \hg, and additional user-defined particles
    \item Output to ROOT or HDF binary formats
\end{itemize}
%
The numerical integration method utilized by PENTrack is a 5th-order Dortmund Prince algorithm~\cite{numerical_recipes}. The pseudo-random generator is the 64-bit Mersenne Twister 19937 algorithm. The seed is generated with a nanosecond resolution time stamp or can be provided directly, which makes PENTrack suitable for running in a multi-threaded mode.

Perhaps most importantly, the PENTrack code base is clearly written and readable, making contributing to it widely accessible. The simulation code is available at [\url{https://github.com/wschreyer/PENTrack}].

%%%%%%%%%%%%%%%%%%%%%%%%%%%%%%%%%%%%%%%%%%

\section{Simulation of UCN rate change from switcher height differential}\label{sec:switcher_height_monte_carlo}

%%%%%%%%%%%%%%%%%%%%%%%%%%%%%%%%%%%%%%%%%%



%%%%%%%%%%%%%%%%%%%%%%%%%%%%%%%%%%%%%%%%%%

\section{Simulation of transport from the UCN source into the apparatus}\label{sec:switcher_wye_transport_monte_carlo}

%%%%%%%%%%%%%%%%%%%%%%%%%%%%%%%%%%%%%%%%%%

Image of PM vector field

Numerical PM vector field + earth's field. Flapper appearing in and out. Pulsed source to emulate beam delivery. Spin depolarization per bounce treated as a fixed percentage and swept. Same for diffusivity

Upper switcher set to stainless steel blankoff vs set to a perfect detector


%%%%%%%%%%%%%%%%%%%%%%%%%%%%%%%%%%%%%%%%%%

\section{1D random walk}\label{sec:1D_random_walk}

%%%%%%%%%%%%%%%%%%%%%%%%%%%%%%%%%%%%%%%%%%

The presence of the Al window in the PM region (Sec.~\ref{sec:PM_description}) results in the loss of some high field seekers from the Al total cross section, absorption from thin oxide films~\cite{pokotilovski_effect_2016}, and bulk elastic scattering from structural inhomogenieties. The effect is compounded when nonspecular bounces cause high field seekers to pass through the Al window region multiple times. We use a simple 1D random walk model to estimate the loss factor from an absorption or effective cross section. The C++ code used to produce the estimate in this section is available at [\url{https://github.com/dougUCN/randomWalk}].

The probability for a neutron to be absorbed in the bulk of the Al window in a single pass can be described using the Beer Lambert Law, Eq.~(\ref{eq:beer_lambert_law})
%
\begin{gather*}
   P_\text{abs}(v) = 1 - \exp \left( - \frac{\ell_\text{window} }{ \lambda_\text{mfp} } \right)
\end{gather*}
%
where $\ell_\text{window}$ is the thickness of the window (\qty{1e-4}{m}), and $\lambda_\text{mfp} = 1 / (N\,\sigma_\text{abs}(v))$ is the mean free path of absorption. $N$ is number density and $\sigma_\text{abs}(v)$ is the velocity-dependent neutron absorption cross section of the material. 

High-field seeking UCN gain approximately 300 neV from the PM field. Assuming UCN in the NiP guide have energies in the \qty{0}{n\eV} to \qty{214}{n\eV} range, high field seekers that encounter the PM window will be in the \qty{300}{n\eV} to \qty{513}{n\eV} regime. $P_\text{abs}(v)$ in this regime can be approximated with a flat line, and we estimate $P_\text{abs}\approx 0.03$ for a single pass through the window.

The UCN path from the source to the cell can be modeled as a 1D random walk, where the step size is the distance traveled by a particle before a nonspecular bounce. From Eqs.~(4.79), (4.70), and (4.48) in Ref.~\cite{golubUCN}, we write
%
\begin{gather}
    d = 2 R\sqrt{\frac{2(2-P_\text{nonspec})}{3P_\text{nonspec}} }
\end{gather}
%
where $R=\qty{1.5}{in}=\qty{0.0381}{m}$ is the radius of the cylindrical UCN guide and $P_\text{nonspec}$ is the probability of a Lambertian (Sec.~\ref{sec:diffuse_reflections}) bounce. Nonspecularity for NiP-coated \acrshort{ss} pipes has previously been measured to be $P_\text{nonspec}\approx0.05$~\cite{pattie_jr_evaluation_2017}, but for Cu or NiP-coated Al, nonspecularity may be treated as a free parameter. For this example calculation we let $P_\text{nonspec}\approx0.05$, giving $d\approx \qty{0.389}{m}$.

NiP has a loss per bounce of $\sim 10^{-4}$~\cite{pattie_jr_evaluation_2017}. The loss per 1D step is then given by
%
\begin{align}
    \text{loss per step} &= 1 - \left(1 - \text{loss per bounce} \right)^{N_\text{spec}}\\
    &\approx N_\text{spec} \times \text{loss per bounce}
\end{align}

When a neutron reaches the end of the 1D beamline, the probability of it entering the cell is dictated by the ratio of the size of the cell entrance to the ``surface area'' of the last 1D step. We model this as
%
\begin{align}
    \text{cell entry chance} &= \left(1 - \left( 1 - \frac{ \text{area of the cell entrance} } {\text{surface area of last 1D step}} \right) ^ {N_\text{spec}} \right)^2 \\
    &= \left(1 - \left( 1 - \frac{ R_\text{entrance}^{2} } {2 R \, d} \right) ^ {N_\text{spec}} \right) ^2\label{eq:cell_entry_chance}
\end{align}
%
where the average number of specular bounces between a nonspecular bounce is given by $N_\text{spec} = 1/P_\text{nonspec}$. For a cell entrance radius $R_\text{entrance}=\qty{1.47}{in}=\qty{0.0373}{m}$ and $N_\text{spec}=20$ we estimate that the chance of entering the cell  is $\approx 0.38$.

Neutrons that enter the cell have a probability of exiting the cell before the end of the filling period. Assuming cell trajectories are isotropic in the cell, this ``exit lifetime'' can be approximated by
%
\begin{gather}
    \text{cell exit lifetime} = \frac{4V_\text{cell}}{\pi R_\text{entrance}^2 \bar{v}}
\end{gather}
%
derived using Eq.~(\ref{eq:effusion-lifetime}). $V_\text{cell}$ is the volume of the precession cell and $\bar{v}$ is the average of velocity of UCN in the cell. For a $v^2\,dv$ distribution (Sec.~\ref{subsec:storageCurves}) and NiP-coated guides (which store UCN up to $\qty{214}{neV}\approx\qty{6.4}{m\per s}$), 
%
\begin{gather}
    \bar{v}=\frac{\int_0^{6.4}{v^3\,dv}}{\int_0^{6.4}{v^2\,dv}}\approx \qty{4.8}{m\per s}
\end{gather}
%
In this example we let the UCN source be located at \qty{0}{m}, and the Al window at \qty{9}{m}, and the cell entrance at \qty{12}{m}. Assuming a preload procedure (Sec.~\ref{subsec:holdingTimeMeasurement}) was followed we let UCN start at the location of a gate valve on the beamline, located at \qty{6}{m}. For a fill time of \qty{50}{s}, UCN with a velocity $\bar{v}$ undergo the 1D random walk until the filling period is over. UCN that return to the D$_2$ source are considered to be lost. UCN that make it to the cell valve location may enter the cell based on Eq.~(\ref{eq:cell_entry_chance}). Neutrons that enter the cell remain for the duration of the cell exit lifetime and exit unless the fill time is completed.

Table~\ref{tb:1D_random_walk} illustrates some example results from the random walk for different single-pass window loss probabilities. To obtain more meaningful results, further studies are required, including the varying of diffusivity, beamline configuration, neutron velocity, and other parameters. 

The improvement of the model also requires comparison with experimental data. Due to the lengthy window removal and installation procedures, measurements on the North beamline in its current configuration without the window have been postponed. With regards to the data in Chap.~\ref{chap:north_beamline_paper}, the alterations in the preload procedure and guide configuration between run conditions was minor enough to allow for the comparisons presented in Sec.~\ref{sec:north_beamline_discussion}, but noticeable enough to obscure the window loss factor.

\begin{table}
\centering
\caption
{Example results from the 1D random walk model (Sec.~\ref{sec:1D_random_walk}). Each simulation uses \qty{1e6}{UCN}. Note that the number of window passes for UCN that make it to the cell must be an odd number}\label{tb:1D_random_walk}
\begin{tabular}{
    S[table-format = 1.2(2)]
    S[table-format = 1.2(2)]
    S[table-format = 1.2(2)]
    S[table-format = 1.2(2)]
}
\toprule
{\makecell{Single pass \\ loss per bounce}} & {UCN in cell} & {UCN lost on window} & {\makecell{Average window passes\\(for UCN in cell)}} \\
\midrule

\bottomrule
\end{tabular}
\end{table}