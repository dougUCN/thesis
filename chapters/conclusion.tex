%%%%%%%%%%%%%%%%%%%%%%%%%%%%%%%%%%%%%%%%%%

\chapter{Conclusions and outlook}\label{chap:conclusion}

%%%%%%%%%%%%%%%%%%%%%%%%%%%%%%%%%%%%%%%%%%

Since successfully demonstrating the Ramsey method of oscillatory fields on stored \acrshort{ucn} with a prototype apparatus in 2017, the \acrshort{lanl} \acrshort{nedm} experiment has continuously undergone iterative improvements. A large \acrshort{msr} with a shielding factor of $10^5$ and residual fields $<\qty{0.5}{nT}$ was assembled and installed. The performance of the North beamline has been characterized, and UCN in excess of \num{40000} have been stored in a prototype chamber, sufficient to achieve a statistical uncertainty of $\sigma_{\gls*{d_n}}=\num{3.32e-27}e\,\text{cm (90\% \acrshort{cl})}$ in 5 calendar years of production running.

The first iteration of the central nEDM apparatus, which consists of the vacuum chamber, electrodes, hydraulic cell valves, and cell wall, has been fabricated and installed in the MSR. Two UCN switchers for UCN transport have been tested and installed on the North beamline. The $B_0$ and MSR feed-through transport coils, essential components of the UCN spin transport and Ramsey method package, have also been installed. We have developed a magnetic impurity scanning system with $\sim \qty{0.1}{nT}$ resolution and scanned all installed experimental components. A simultaneous spin analyzer was fabricated and tested with polarized UCN. It has since been upgraded, and a second simultaneous spin analyzer has been built with the improved design. Since 2020, a high resolution \qty{250}{MHz} \acrshort{adc} has been used and tested with UCN signals from both \acrshort{sipm}s and \acrshort{pmt}s. Early iterations of a central experiment control system have been used, and a proof-of-concept control system hosted on the internal LANL network has been created.

%%%%%%%%%%%%%%%%%%%%%%%%%%%%%%%%%%%%%%%%%%

\section{Next steps}

%%%%%%%%%%%%%%%%%%%%%%%%%%%%%%%%%%%%%%%%%%

Several immediate fixes and improvements are on the way. MSR degaussing loop contacts will be fixed by summer 2023. With the room properly degaussed, a newly constructed prototype magnetic field mapper can accurately characterize residual magnetic fields and the performance of the $B_0$ coil. Field mapping of the MSR feed-through transport coils is in progress. 

The cell valve actuation arm that was previously impeded from access to the top cell has been redesigned and constructed. An in-situ cell valve monitoring method to determine the status of the valve is being implemented, where a gauge reads the pressure in the hydradulic line. When the piston outside the MSR exerts pressure on a fully closed cell valve, assuming there is no leak in the line, the pressure spikes well above baseline and can be used as feedback to the control system or during the calibration process.

We have designed a series of measurements for the 2023 cycle with the goal of troubleshooting issues regarding UCN density and spin asymmetry. The measurements involve extending the beamline until the UCN guides extend out through the back of the MSR. This allows for flow-through measurements to evaluate UCN transport and UCN spin transport without the need for storage. With the proven reliability of the single arm spin analyzer, spin flipper, and UCN detector system (drop detector), a second drop detector is being built for the 2023 measurements. Both upper and lower beamlines extending from the switchers through the MSR will have drop detectors attached. 

The holding coils along the North beamline have been reconfigured and are being mapped for possible field zeros. A second version of the switcher wye is being manufactured in the event that the first version was a source of depolarization.

A new set of \acrshort{hv} electrodes has been fabricated and will be \acrshort{dlc}-coated. A HV feed through with nonmagnetic components has been designed, with a goal of applying HV by the end of 2023. The magnetic impurity scanner will also undergo further improvements, with the largest priority being the reduction of ambient magnetic background noise and drif. Its performance is currently being benchmarked with dipoles of known strengths (i.e. current loops on the turntable).

Prototypes of Cs external magnetometers produced by industry collaborators have been tested in the MSR and are being optimized. Development of the \hg comagnetometer and external \hg cells is ongoing, and assembly of the laser system that probes the \hg cells is underway.
