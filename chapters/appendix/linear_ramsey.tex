%%%%%%%%%%%%%%%%%%%%%%%%%%%

\chapter{Numerical generation of a Ramsey fringe}\label{appx:ramsey_numerical}

%%%%%%%%%%%%%%%%%%%%%%%%%%%

In this appendix we include C++ code for numerical generation of a Ramsey fringe with linear RF, as utilized in Chap.~\ref{chap:spinManipulation}. The repository located at [\url{https://github.com/dougUCN/ramseyCPP}] contains additional documentation, example Rabi sequences and Bloch Siegert shift calculations, and a CMake file. 

First, we summarize numerical evaluation of equations from Chap.~\ref{chap:spinManipulation} for \ucn spin motion in static and time dependent magnetic fields. The function \mintinline[bgcolor=white, style=sas, fontsize=\small]{cpp}{void neutron::larmorPrecess} is based on the analytical solution for Larmor precession, Eq.~(\ref{eq:larmor_solution}), with the substitution
%
\begin{align}
    a &= \text{Re}(a)+i\text{Im}(a)\label{eq:a0_Re_Im}\\
    b &= \text{Re}(b)+i\text{Im}(b)\label{eq:b0_Re_Im}
\end{align}
%
which gives 
%
\begin{align}
    \text{Re}(a(t)) &= \text{Re}(a)\cos\left(\frac{i\omega_0 t}{2}\right)+\text{Im}(a)\sin\left(\frac{i\omega_0 t}{2}\right)\\
    \text{Im}(a(t)) &= \text{Im}(a)\cos\left(\frac{i\omega_0 t}{2}\right)-\text{Re}(a)\sin\left(\frac{i\omega_0 t}{2}\right)\\
    \text{Re}(b(t)) &= \text{Re}(b)\cos\left(\frac{i\omega_0 t}{2}\right)-\text{Im}(b)\sin\left(\frac{i\omega_0 t}{2}\right)\\
    \text{Im}(b(t)) &= \text{Im}(b)\cos\left(\frac{i\omega_0 t}{2}\right)+\text{Re}(b)\sin\left(\frac{i\omega_0 t}{2}\right)
\end{align}
%
Similarly, the function \mintinline[bgcolor=white, style=sas, fontsize=\small]{cpp}{vector<double> neutron::derivs} describes spinor evolution in a Rabi sequence with either linear or circular RFs. For the linear case, combining Eqs.~(\ref{eq:rabi_linear_1})--(\ref{eq:rabi_linear_2}) with (\ref{eq:a0_Re_Im})--(\ref{eq:b0_Re_Im}) gives
%
\begin{align}
    \text{Re}(\dot{a}) &= \frac{1}{2}\left[\omega_0\text{Im}(a) +\omega_\ell\cos(\omega t +\phi)\text{Im}(b) \right] \\
    \text{Im}(\dot{a}) &= \frac{1}{2}\left[-\omega_0\text{Re}(a)-\omega_\ell\cos(\omega t+\phi)\text{Re}(b) \right] \\
    \text{Re}(\dot{b}) &= \frac{1}{2}\left[-\omega_0\text{Im}(b)-\omega_\ell\cos(\omega t+\phi)\text{Im}(a) \right] \\
    \text{Im}(\dot{b}) &= \frac{1}{2}\left[\omega_0\text{Re}(b) +\omega_\ell\cos(\omega t +\phi)\text{Re}(a) \right]
\end{align}
%
This can be expanded to the circular RF case, by numerically integrating  Eqs.~(\ref{eq:rabi_2}) and (\ref{eq:rabi_3}). Following the notation of this appendix, this is written as
%
\begin{align}
    \text{Re}(\dot{a}) &= \frac{1}{2}\left[\omega_0\text{Im}(a) +\omega_\ell\cos(\omega t +\phi)\text{Im}(b) -\omega_\ell\sin(\omega t +\phi)\text{Re}(b) \right]\\
    \text{Im}(\dot{a}) &= \frac{1}{2}\left[-\omega_0\text{Re}(a)-\omega_\ell\cos(\omega t+\phi)\text{Re}(b) -\omega_\ell\sin(\omega t+\phi)\text{Im}(b) \right] \\
    \text{Re}(\dot{b}) &= \frac{1}{2}\left[-\omega_0\text{Im}(b)-\omega_\ell\cos(\omega t+\phi)\text{Im}(a) +\omega_\ell\sin(\omega t+\phi)\text{Re}(a) \right] \\
    \text{Im}(\dot{b}) &= \frac{1}{2}\left[\omega_0\text{Re}(b) +\omega_\ell\cos(\omega t +\phi)\text{Re}(a) +\omega_\ell\sin(\omega t+\phi)\text{Im}(a) \right]
\end{align}
%
Coupled ODEs are solved with a 4th-order Runge-Kutta integration method, which is described in Ref.~\cite{numerical_recipes}. Integration accuracy may be determined via comparison to analytical solutions, such as Eq.~\ref{eq:rabi_probZ} for a Rabi $\pi$ flip using a circular RF, or by the normalization requirement of the spinor $\Braket{\psi(t) | \psi(t)} - 1 = 0$.

\section{ramsey.cpp}

\inputminted{cpp}{code_snippets/ramsey.cpp}

\section{neutron.hpp}

\inputminted{cpp}{code_snippets/neutron.hpp}

\section{neutron.cpp}

\inputminted{cpp}{code_snippets/neutron.cpp}
