%%%%%%%%%%%%%%%%%%%%%%%%%%%

\chapter{
    \texorpdfstring{Derivation of $R=\frac{1}{4}\langle v \rangle \rho A$}
    {R=1/4 v rho A}\label{appx:ucn_effusion}
}
%%%%%%%%%%%%%%%%%%%%%%%%%%%

The rate of detected \ucn is often related to density in a volume by the equation
%
\begin{gather}
    R=\frac{1}{4}\langle v \rangle \rho A\label{eq:effusion-appendix}
\end{gather}
%
where $\rho$ is the neutron density, $A$ is the area of the detector, and  $\langle v \rangle$ is the average velocity of the neutron population. The derivation for Eq.~(\ref{eq:effusion-appendix}) can be found in any statistical mechanics textbook discussion on effusion, but is reproduced here for convenience.

Assume an isotropic velocity distribution, that $A$ is small relative to the mean free path of \ucn in the volume, and that when a \ucn is detected it exits the system.  We write
%
\begin{align}
    R &=\Phi \cdot A = \text{(\# of particles exiting per unit time per unit area)} \cdot A \nonumber \\
    &= A \rho \int^\infty_{-\infty}dv_x \, \int^\infty_{-\infty}dv_y \, \int^\infty_{0} f(\vec{v})v_z
\end{align}
%
for some velocity distribution $f(\vec{v})$. Due to isotropy, $f(\vec{v})=f(v)$. Let the detector be oriented along the $\vec{z}$ axis. The $dv_z$ integral has limits $[0,\infty]$ because the detector only counts particles moving towards it.

We change to spherical coordinates, setting $v_z = v\cos \theta$. This gives
%
\begin{align}
    R &= A \rho \int^{2\pi}_0 d\phi \, \int^{\pi/2}_0 d\theta \, \int^\infty_0 dv \, v^2 f(v) v \cos \theta \sin \theta \label{eq:effusion-derive-1} \\
    &= A \rho \pi \int^\infty_0 dv \, v^3 f(v) \label{eq:effusion-derive-2} \\
    &= \frac{A\rho}{4} \int^\infty_0 dv \, \tilde{f}(v)v \label{eq:effusion-derive-3} \\
    &\equiv \frac{1}{4}\langle v \rangle \rho A
\end{align}
%
where the limits of integration from $\theta$ in Eq.~(\ref{eq:effusion-derive-1}) are again set such that only particles moving towards the detector are counted. To get from Eq.~(\ref{eq:effusion-derive-2}) to Eq.~(\ref{eq:effusion-derive-3}) we use the definition of the speed particle distribution function
%
\begin{align}
    \tilde{f}(v) &= \int^\pi_0 d\theta \int^{2\pi}_0 d\phi \, \tilde{f}(v,\theta,\phi) \\
    &= \int^\pi_0 d\theta \int^{2\pi}_0 d\phi \, f(v)v^2\sin \theta = 4\pi v^2 f(v)
\end{align}
